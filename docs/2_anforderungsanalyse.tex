\section{Anforderungsanalyse}\label{kapitel1}
\subsection{Allgemein}
Die App soll es ermöglichen, einen Lauf mitzuschneiden, um ihn anschließend an eine Gruppe an Freunden zu veröffentlichen. Diesen ist es dann möglich, mit ihren Freunden zu laufen bzw. gegen diese anzutreten. 

Um diese Funktionalität zu ermöglichen sind das Schreiben einer Android-App und das Aufzetzen eines Webserver nötig.

\subsection{Webserver}

Der Webserver stellt eine Schnittstelle bereit, die einem Klienten nach der Authentifizierung folgende Funktionalitäten bereitstellt:
\begin{itemize}
\item Nutzer abzufragen, anzulegen und zu löschen,
\item Läufe abzufragen, anzulegen und zu löschen,
\item Gruppen abzufragen, anzulegen und zu löschen,
\item Nutzer zu Gruppen hinzuzufügen oder zu entfernen,
\item Freunde einzuladen und
\item Kacheln für Höhendaten anzufordern und herunterzuladen.
\end{itemize}
\vspace{\baselineskip}

Hierbei werden folgende Anforderungen gestellt:
\begin{itemize}
\item Die Authentifizierung muss sicher sein, sodass das Passwort des Nutzers und seine persönlichen Daten sicher vor Fremdzugriff und Manipulation durch Unbefugte sind.
\item Der Server muss innerhalb von 5s Sekunden reagieren.
\item Mobile Datenbeschränkungen müssen bedacht werden, sowohl in Bezug auf Geschwindigkeit als auch Datenmengen.
\end{itemize}

\subsection{Android-App}
Die Android-App soll dem Nutzer folgenden Funktionalitäten bereitstellen:
\begin{itemize}
\item Anmeldung beim Webserver
\item Registrierung beim Webserver
\item Abmeldung vom Webserver
\item Laufen ohne währenddessen gegen Freunde anzutreten
\item Laufen um währenddessen die Position der Freunden mitverfolgen zu können
\item Eintreten in Gruppen, Austreten aus Gruppen und Erstellen von Gruppen
\item Einladen appfremder Nutzer in eine Gruppe
\item Einladen anderer Nutzer der App in eine Gruppe
\item Weitergabe des Administratorprivilegs innerhalb der Gruppe durch den Gruppenadministrator
\end{itemize}

\subsection{Usability}
Die App muss einfach zu bedienen sein. Wir richten uns allgemein an Sportler und Sportinteressierte. Außer grundlegenden Kenntnissen zur Nutzung von Android-Apps können wir keine Annahmen zu den technischen Fähigkeiten des Nutzers machen.

Teile der App sind für die Nutzung beim Joggen. Auch hierbei muss es dem Nutzer möglich sein relevante Informationen zu erkennen und sicher mit der App zu interagieren.

\subsection{Sicherheit}
Um Nutzer zu authentifizieren ist eine Nutzername/Passwort-Vergabe notwendig. Diese müssen ausreichend gesichert sein, um unauthorisierte Zugriffe auf Ressourcen zu verhindern. 

Außerdem muss die Sicherheit des Gerätes zu jedem Zeitpunkt gewährleistet sein, d.h. die Einschleusung über die App von Schadcode darf nicht möglich sein.

